% Options for packages loaded elsewhere
\PassOptionsToPackage{unicode}{hyperref}
\PassOptionsToPackage{hyphens}{url}
%
\documentclass[
]{article}
\usepackage{lmodern}
\usepackage{amssymb,amsmath}
\usepackage{ifxetex,ifluatex}
\ifnum 0\ifxetex 1\fi\ifluatex 1\fi=0 % if pdftex
  \usepackage[T1]{fontenc}
  \usepackage[utf8]{inputenc}
  \usepackage{textcomp} % provide euro and other symbols
\else % if luatex or xetex
  \usepackage{unicode-math}
  \defaultfontfeatures{Scale=MatchLowercase}
  \defaultfontfeatures[\rmfamily]{Ligatures=TeX,Scale=1}
\fi
% Use upquote if available, for straight quotes in verbatim environments
\IfFileExists{upquote.sty}{\usepackage{upquote}}{}
\IfFileExists{microtype.sty}{% use microtype if available
  \usepackage[]{microtype}
  \UseMicrotypeSet[protrusion]{basicmath} % disable protrusion for tt fonts
}{}
\makeatletter
\@ifundefined{KOMAClassName}{% if non-KOMA class
  \IfFileExists{parskip.sty}{%
    \usepackage{parskip}
  }{% else
    \setlength{\parindent}{0pt}
    \setlength{\parskip}{6pt plus 2pt minus 1pt}}
}{% if KOMA class
  \KOMAoptions{parskip=half}}
\makeatother
\usepackage{xcolor}
\IfFileExists{xurl.sty}{\usepackage{xurl}}{} % add URL line breaks if available
\IfFileExists{bookmark.sty}{\usepackage{bookmark}}{\usepackage{hyperref}}
\hypersetup{
  pdftitle={Resume},
  hidelinks,
  pdfcreator={LaTeX via pandoc}}
\urlstyle{same} % disable monospaced font for URLs
\setlength{\emergencystretch}{3em} % prevent overfull lines
\providecommand{\tightlist}{%
  \setlength{\itemsep}{0pt}\setlength{\parskip}{0pt}}
\setcounter{secnumdepth}{-\maxdimen} % remove section numbering

\title{Resume}
\usepackage{etoolbox}
\makeatletter
\providecommand{\subtitle}[1]{% add subtitle to \maketitle
  \apptocmd{\@title}{\par {\large #1 \par}}{}{}
}
\makeatother
\subtitle{Summary of my professional career}
\date{}

\begin{document}
\maketitle

\begin{center}\rule{0.5\linewidth}{\linethickness}\end{center}

\hypertarget{summary}{%
\section{Summary}\label{summary}}

\begin{center}\rule{0.5\linewidth}{\linethickness}\end{center}

\begin{itemize}
\item
  Strong project management skills as demonstrated by 3 successful
  projects in industrial sectors, biomedical, information and spatial,
  leading to 6 publications in advanced scientific journals (IEEE Xplore
  Digital Library)
\item
  High-level technical skills in applied mathematics, machine learning,
  statistics, to unlock information in the structured/unstructured data
  (signal and image) and deliver valuable insights to provide strong
  imaging system knowledge, and passion about the development of
  algorithms in machine learning,
\item
  Excellent oral and written communication skills resulting in 7
  first-author papers and 3 talks
\end{itemize}

\hypertarget{fields-of-expertise}{%
\subsubsection{Fields of expertise}\label{fields-of-expertise}}

\begin{itemize}
\tightlist
\item
  Images and Signals Processing
\item
  Artificial Intelligence \& Machine Learning
\item
  Neural Networks / Deep learning
\item
  Research and Development (R\&D)
\item
  Project Management
\item
  Statistics and Bayesian Inference
\item
  Imaging system knowledge
\end{itemize}

\hypertarget{work-experience}{%
\section{Work Experience}\label{work-experience}}

\begin{center}\rule{0.5\linewidth}{\linethickness}\end{center}

\hypertarget{rd-engineer-in-image-processing}{%
\subsubsection{R\&D Engineer in Image
Processing}\label{rd-engineer-in-image-processing}}

\begin{center}\rule{0.5\linewidth}{\linethickness}\end{center}

\href{http://www.l2s.centralesupelec.fr/}{Signal and Systems Laboratory
(L2S)}, Centrale Supélec \& \href{http://www.ias.u-psud.fr/}{Spatial
Astrophysics Institute (IAS)}, CNRS, France.

10 / 2015 - 11 / 2018

\textbf{Description of the Project:} Hyperspectral object reconstruction
from multi-spectral data observed by an infrared imager, aboard NASA's
next JWST space telescope. Launch of the telescope in 2021.

\textbf{Accomplishments}\\
► Strong project management skills as evidenced by the L2S - IAS
collaboration on the hyperspectral data reconstruction for the next
space mission JWST (NASA/ESA/CSA), leading to the innovation of 2 new
algorithms from scratch for the processing of infrared images using
applied mathematics and machine learning in Python

► Excellent leadership skills and experience mentoring as demonstrated
by teaching ``signal processing'' to +60 engineering students at Ecole
Polytech Paris-Sud resulting in a 100\% success rate

► Gained advanced communication skills through participation at 2
conferences, 1 workshop, and 1 seminar, resulting in 3 talks in front of
+100 experts, publication of 4 papers as a first author and 1 poster
presentation

► High-level technical skills, including project roadmap definition,
literature review, system modeling, implemented of various optimization
algorithms

\textbf{Publication in peer review international conferences}\\
► GRETSI 2017\\
► EUSIPCO 2017\\
► EUSIPCO 2018

\textbf{Public Talks}\\
► GDR-ISIS, Paris, France.\\
► EUSIPCO 2018, Rome, Italy.\\
► EUSIPCO 2017, Kos, Grece.

\textbf{Technical environment}\\
► Linux Ubuntu 64-bit\\
► Python: Numpy, Scikit-learn, Matplotlib, Pandas

\hypertarget{teaching-assistant-in-signal-processing}{%
\subsubsection{Teaching Assistant in Signal
Processing}\label{teaching-assistant-in-signal-processing}}

\begin{center}\rule{0.5\linewidth}{\linethickness}\end{center}

\href{http://www.polytech.u-psud.fr/fr/formations/electronique-et-systemes-robotises.html}{Ecole
Polytech Paris-Sud}, University of Paris-Sud, France.

2016-2018

\textbf{Description of the project}: I was in charge of teaching two
classes of total number of +60 under graduate students at the
engineering school \emph{Polytech Paris-Sud}.

\textbf{Accomplishments}\\
► Teaching of Signal Processing\\
► Supervision of two groups of +60 engineering students\\
► Animation of tutorials\\
► Student assessment and exam preparation\\
► Teaching contents: Fourier Transform, Linear filtering, Sampling,
Applied mathematics, Deterministic random signal processing, basis in
statistics, \ldots{}\\
► Teaching students to perform signal processing in \emph{MATLAB}

\hypertarget{image-processing-engineer---computer-vision}{%
\subsubsection{Image Processing Engineer - Computer
Vision}\label{image-processing-engineer---computer-vision}}

\begin{center}\rule{0.5\linewidth}{\linethickness}\end{center}

\href{http://www.synchromedia.ca/}{Synchromedia Laboratory}, Montreal,
Quebec, Canada

02/2015-08/2015

\textbf{Description of the project}: coming soon\ldots{}

\textbf{Accomplishments}\\
► Demonstrated flexibility skills through the development of image
processing algorithms for the preservation of historical documents,
resulting in a new method for processing document images using a linear
mixing model in Matlab

► Improved project and time management skills by defining the problem
and the project roadmap, Implementing image processing algorithms in
Matlab

► Developed communication skills through writing and presenting the
project report in English

\textbf{Technical environment}\\
► Linux Ubuntu 64-bit\\
► MATLAB

\hypertarget{signal-and-image-processing-engineer}{%
\subsubsection{Signal and Image Processing
Engineer}\label{signal-and-image-processing-engineer}}

\begin{center}\rule{0.5\linewidth}{\linethickness}\end{center}

\href{https://www.math.u-bordeaux.fr/imb/spip.php}{The Mathematical
Institue of Bordeaux}, France.

02/2014 - 07/2014

\textbf{Description of the project}: \ldots{}

\textbf{Accomplishments}\\
► Gained high-level technical skills through the implementation from
scratch of signal and image restoration methods using advanced applied
mathematics, and the implementation of optimization algorithms in Matlab

► Successful time management skills as demonstrated by solving, in a
short time, three targeted problems (deconvolution, denoising,
inpainting), leading to an efficient algorithm for multiple applications

\textbf{Technical environment}\\
► Linux Ubuntu 64-bit\\
► MATLAB

\hypertarget{machine-learning-engineer---rd}{%
\subsubsection{Machine Learning Engineer -
R\&D}\label{machine-learning-engineer---rd}}

\begin{center}\rule{0.5\linewidth}{\linethickness}\end{center}

Biomedical laboratory, USTHB. Algeria.

02/2012 - 02/2013

\textbf{Description of the project}: As part of the internship at the
end of engineering studies, I held the position of machine learning
engineer for the EEG biomedical signal epilepsy detection project. The
goal is to anticipate the treatment of the pathology, and protect the
individual suffering from the pathology in case of epileptic seizure.

\textbf{Accomplishments}\\
► Developed project management skills through the development of a
decision support tool for epilepsy detection from an EEG signals
database using machine learning, resulting in the proposition of a new
pipeline for the classification of biomedical signals in Matlab

► Improved writing and communication skills through writing papers,
leading to 2 publications in international conferences

► Ability to successfully handle a research project in machine learning
through the definition of the roadmap project, the study of
state-of-the-art works, data analysis, applied mathematics, signal
processing

\textbf{Technical environment}\\
► MATLAB

\textbf{Publication in peer review international conferences}\\
► CISTEM'2018\\
► WoSSPA'2013

\begin{center}\rule{0.5\linewidth}{\linethickness}\end{center}

\hypertarget{education}{%
\section{Education}\label{education}}

\begin{center}\rule{0.5\linewidth}{\linethickness}\end{center}

\hypertarget{university-of-paris-saclay}{%
\subsubsection{University of
Paris-Saclay}\label{university-of-paris-saclay}}

\begin{center}\rule{0.5\linewidth}{\linethickness}\end{center}

Doctor of Philosophy (PhD), Signal and Image Processing.\\
2015 - 2018

\textbf{Publications in peew review national/international conference}\\
► 28th and 28th EUSIPCO (European Signal and Image Processing
Conference).\\
► 27th and 28th EUSIPCO (European Signal and Image Processing
Conference).\\
► 25th Colloque GRETSI (Groupe d'Etudes du Traitement du Signal et des
Images).\\
► Co-author of an article published in Astronomy \& Astrophysics
Journal.

\hypertarget{university-of-bordeaux}{%
\subsubsection{University of Bordeaux}\label{university-of-bordeaux}}

\begin{center}\rule{0.5\linewidth}{\linethickness}\end{center}

Master's degree, Signal and Image Processing.\\
2013 - 2015

► Master 2: 2014-2015 (Ranked : 2/21)\\
► Master 1: 2013-2014

\hypertarget{ecole-polytechnique}{%
\subsubsection{Ecole Polytechnique}\label{ecole-polytechnique}}

\begin{center}\rule{0.5\linewidth}{\linethickness}\end{center}

Engineer degree, Electrical engineering.\\
2007 - 2012

► 2007-2009: Fundamental Engineering\\
► 2009-2012: Electrical and Electronics Engineering

\textbf{Publication in peer review international conference} * WOSSPA
2013 (Workshop on Systems, Signal Processing, and their Applications)

\begin{center}\rule{0.5\linewidth}{\linethickness}\end{center}

\hypertarget{techniques-software-instrumentation}{%
\section{Techniques, Software \&
Instrumentation}\label{techniques-software-instrumentation}}

\begin{center}\rule{0.5\linewidth}{\linethickness}\end{center}

\textbf{Programming Languages}: Python, Matlab, Latex, Markdown, RST

\textbf{Libraries}: OpenCV, Numpy, Matplotlib, scrikit-learn, PyTorch,
TensorFlow, Keras

\textbf{IDE}: VS Code, JupyterLab, Spyder

\textbf{Tools}: Linux OS, Gitlab, Docker, bash, Slack

\begin{center}\rule{0.5\linewidth}{\linethickness}\end{center}

\hypertarget{affiliations-hobbies}{%
\section{Affiliations \&/ Hobbies}\label{affiliations-hobbies}}

\begin{center}\rule{0.5\linewidth}{\linethickness}\end{center}

Running, Badminton, Reading, Writing

\end{document}
